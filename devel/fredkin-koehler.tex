\documentclass{article}
\newcommand{\Laplace}{\Delta}

\begin{document}



{\large On the Fredkin-Koehler Finite Element / Boundary Element Method}

\section{Introduction}

Applications of field theory to systems with complicated geometries
are common especially in the engineering disciplines. The usual
approach is to choose a finite selection of field degrees of freedom
such that the truncation of the physics to these degrees of freedom
gives a sufficiently accurate quantitative description of the
behaviour of the system which is computationally manageable.

Usually, the finitely many degrees of freedom to which the system is
truncated are chosen in such a way that they model spatially localized
contributions to fields. A widespread approach is to decompose space
into a mesh of polyhedral cells and associate a set of support points
with these cells. Base functions usually are then chosen in such a way
that they satisfy certain smoothness requirements, have compact
support, and have value one at a single support point and zero at all
others. (Note that it is easy to relax some of these requirements
which leads to generalizations that also include meshless methods.)

One of the nice aspects of this `Finite Element Method' is that
polyhedral decompositions allow one to model most material boundaries
quite nicely if the mesh is chosen in such a way that material
boundaries are approximated by cell boundaries.

There are situations where all the interesting physical information is
contained in just a small fraction of the numerical values that are
needed to describe the system in a finite element approach. Evidently,
it would be very desirable to maximize the ratio of actually useful
data to additional bookkeeping data. One particularly interesting case
is the simulation of magnetzation dynamics. As magnetic moments
interact over large distances via the magnetic fields which they
carry, one might consider it necessary to mesh-discretize a large
enough portion of space which not only contains all the magnetic
bodies, but also enough of the surrounding space so that all magnetic
fields can `spread out' without running into artificial boundary
conditions that were imposed by choosing too small a simulation
region. However, as the most interesting questions one would like to
ask about such a system do not involve the fields outside the magnetic
bodies, but only the dynamics of the magnetization, it would be highly
desirable to get rid of the necessity to compute the magnetic field
throughout the surrounding space. This would correspond to
`integrating out' the `outer space degrees of freedom', leaving us
with a densely occupied coupling matrix that contains all the
information about how the surface magnetization at one point of the
material surface influences the magnetic field at some other point of
the surface.

\section{The Boundary Element Matrix}

For the sake of definiteness, we will constrain the subsequent
discussion to the case of unstructured simplicial (tetrahedral) meshes
in three dimensions and use common electrodynamical terminology --
even though it may be considered inferior to the differential forms
approach from both the conceptual and aesthetic point of view.

In the absence of currents, the demagnetization field $\vec H_{\rm
demag}$ is a rotation-free vector field whose sources (divergence)
just cancel the divergence of the magnetization, such that
\begin{equation}
\vec{\nabla} \cdot \vec{B} = \vec{\nabla}\left(\vec{H}_{\rm demag}+4\pi\vec{M}\right)=0.
\end{equation}


So, if we mesh-discretize the magnetic bodies as well as a large chunk
of the surrounding space, we can compute $\vec H_{\rm demag}$ from the
gradient of a magnetic scalar potential $\phi_m$ -- since $\vec H_{\rm
demag}$ is rotation-free. The divergence of $\vec H_{\rm demag}$ --
which hence is just $\Laplace\phi_m$ -- must then cancel the
divergence of $4\pi\,\vec M$. Talking in wave-number ($\vec k$-)space
language, $\vec H_{\rm demag}$ serves the purpose to remove to remove
the longitudinally polarized components of $\vec M$. We can hence
obtain $\vec H_{\rm demag}$ by the following procedure:
%
%\begin{equation}

\begin{tabular}{ll}
1. & Compute $\rho_m = \vec\nabla\cdot\vec M$\\
2. & Solve $\Laplace\phi_m = \rho_m$ with zero Dirichlet boundary conditions far away.\\
3. & Compute $H_{\rm demag}=\vec\nabla \phi_m$.
\end{tabular}

%\end{equation}
%
This is easily implemented using the finite element method. However,
some care has to be taken in order to get the surface divergencies
from the boundaries of magnetic materials right.

Evidently, step (1) and (3) above can be constrained to the magnetic
material and its boundary. What can we do in order to also constrain
the computation of the magnetic scalar potential -- step (2) -- to
that region? The key to that question is a simple observation from
potential theory: if we take a surface $\Omega$, then a surface charge
density located at $\Omega$ will cause a kink in the (continuous)
potential, that is, a jump in the derivative. A surface dipole
density, however, can be regarded as the limit of the distribution of
charge densities of opposite magnitude over two surfaces $\Omega^+$
and $\Omega^-$ close to $\Omega$, where the limit is taken in such a
way that at the same time as the separation of the $\Omega^\pm$
surfaces is taken to zero and these surfaces are made to coincide with
$\Omega$, the charge densities are increased just in the same way. In
effect, we get a kink in the potential at $\Omega^+$ which is
compensated by an anti-kink at $\Omega^-$, so that the derivatives of
the potential (and hence the field strengths) to the left and to the
right of the dipole layer are the same. We do however get a jump in
the potential -- the smaller we make the gap, the steeper the
potential changes inside the gap, so that the total magnitude of the
jump remains constant.

[Diagrams!]

This means that we can associate the effects of jumps in the potential
to surface dipole densities. In other words, if we decompose a
continuous potential into two pieces $\phi=\phi_1+\phi_2$, such that
each of these pieces has jumps, we can interpret these jumps as the
effects of two artificially introduced opposite (cancelling) surface
dipole densities. The idea now is to decompose the magnetic potential
$\phi_m$ into a piece $\phi_1$ whose sources are given by the
divergencies and surface divergencies inside the magnetic material,
and whic jumps to zero outside, and a remainder $\phi_2$, which has
the opposite jump at the magnetic material boundaries, and is
source-free everywhere, up to the dipole density responsible for the
jump.

Once we found an arbitrary solution for the inhomogeneous potential
equation governing $\phi_1$ inside the magnetic region (which may be
determined only up to the addition of a source-free (homogeneus)
solution(???)), we can then take the jump of $\phi_1$ from its value
at the inner side of the material boundaries to zero at the outer side
as dipole occupation density source and compute $\phi_2$ at an
arbitrary point in space by integrating the field of a dipole source
distribution over the surface. The field of a localized dipole source
of unit strength is just given by the directional derivative of the
Green's function, or propagator, $G(x,y)=1/|\vec x-\vec y|$.

As it would be costly to determine $\phi_2$ in this way all over the
magnetic body, we rather make use of the property $\Laplace \phi_2=0$
and determine $\phi_2$ just at the magnetic material boundary, and
solve the Laplace equation with Dirichlet boundary conditions for the
interior.

XXX what about the ($\Omega(x)/4pi-1)\phi_1$ term?

XXX we get phi2 as a cofield?!?

XXX factors 4pi are not yet right everywhere?!

\section{Some random notes on the Fredkin-Koehler formula for $\phi2$}

\begin{itemize}

\item $\phi_2$ is just a function of $\phi_1$ -- therefore,
we do not expect any problems concerning forgotten normalization factors
or differences in conventions on factors of $4\pi$.

\item The first term is just the potential of a dipole surface at some point
outside the surface, which furthermore is constructed in such a way that it
vanishes at infinity. The ``minus-one'' term implements the surface jump
in the potential by $\phi_1$ (that's what $\phi_2$ is supposed to be good for!)

\item note that $\phi_1+\phi_2$ is continuous at the boundary. $\phi_1$ will have
zero derivative at the boundary due to open Neumann B.C.s.

\item the $\Omega/4\pi$ term is the jump across the surface.
One should try to make this plausible by first considering a flat surface.
The potential induced by a surface homogeneously occupied with dipoles is
just proportional to the spatial angle under which this surface appears.
If we come close to the surface, the immediate vicinity of the surface point
directly under the observer occupies the whole Gesichtsfeld(translation?),
so this gives us the local contribution to the potential.

\item\dots but this means that we {\em must not} try to include the diagonal part from
the $\partial G/\partial n$ term. If we did, we would have this contribution twice!

\end{itemize}


\end{document}
